%!TEX program = xelatex
\documentclass[a4paper,UTF8]{ctexart}

\usepackage{fontspec}
\usepackage{amsmath}
\usepackage{float}
\usepackage{graphicx}
\usepackage{amssymb}
\usepackage{epstopdf}
\usepackage{mathrsfs}
\usepackage{color}
\usepackage{listings}
\usepackage{ulem}
\usepackage{lmodern}
\usepackage{fontenc}
\usepackage[margin=2cm]{geometry}
\usepackage{titlesec}
\usepackage{tikz}
\usepackage{hyperref}
\usepackage[noblocks]{authblk}
\usepackage{MyTitle}

\setlength{\parskip}{5pt}

\titleformat{\section}[hang]{\bfseries\fontsize{16pt}{0pt}\selectfont}{\thesection}{.5em}{}{}


% 对于使用 Tex Pad 打开本 TeX 文档的用户,请使用这行代码
\usepackage[cache=false,outputdir=.texpadtmp]{minted}
% 其它用户使用这行
%\usepackage{minted}
%



%To do: 修改标题和作者及联系方式
\title{HITCON2015 Quals, Reverse400}

\author{封寒松(roy)}
\affil{哈尔滨工业大学,计算机科学与技术学院,royfhs@163.com}

\date{}

\begin{document}
\maketitle

%-------------------------------------s-----------------------------------------------------------------------------------------
%------------------------------------------------------------------------------------------------------------------------------
%To do: 正文开始
%------------------------------------------------------------------------------------------------------------------------------
%------------------------------------------------------------------------------------------------------------------------------

\section{Description}

\begin{quizdesc}[label=Reverse400 NPC]
NPC says "I am NPC."
npc-4bc7ebfe94c8fdc93832bc0e7af1279b
\end{quizdesc}
  
\section{Hint}

Check the black/white connected components in 2D grid of cells.

\section{Reversing}

The binary has a long main function which appears to be optimized and inlined C++. I used the Hex-Rays decompiler in IDA Pro to decompile it.

    There are several stages in the function.
    
  \subsection{Length check}
  
  The binary calls sub402980 to check the validity of the input length. This function does a bunch of somewhatnasty math which I don't want to reverse. Instead, I take advantage of the fact that the function only proceeds to memset if the length check succeeds:

\begin{minted}{bash}
for ((i=1; i<100; i++));
    do echo $i;
    ltrace -ff ./npc
    $(python -c "print 'hitcon{' + 'A'*$i + '}'") 2>&1
    grep memset;
done
\end{minted}
      
      This shows that memset is only called on an input length of 80.
      
\subsection{Input conversation}
     At \code{0x400c2a}, the binary then sets up some kind of table of size \code{127}, inserting \code{0x20} values into the table at
specific indices. It then runs through the input byte-by-byte, indexing into this table using the ASCII value of
each input byte. I dumped the table using GDB, and picked a byte corresponding to a valid value \code{'A'}.

     Using qira we can watch this function fill a buffer bit-by-bit. The buffer is filled with 50 bytes of binary data from our 80-byte input, so we guess that each input character is converted into 5 output bits in a base32-like fashion.

\subsection{String unpacking}

    Next, at \code{0x400ef5}, the binary fills out some kind of array (a C++ vector) using a C++ string (which is
initialized from the init function \code{0x401e60}). Again, from laziness, we can just ask GDB to tell us what the
vector contains after the function finishes. The vector contains 400 integers. Most are zero, with a few nonzeros
between 1 and 8. We call this vector the "map".

  \subsection{First loop}

At \code{0x40104a}, the binary calls \code{sub402490} which sets up two more C++ vectors: one containing the integers 0 to 400,
and the other containing 400 1s.

It then goes into a triple loop at \code{0x401139} which has two nested loop variables going from 0-19. It now becomes
clear that the program is operating on a 20x20 grid, and the bitwise operations on the buffer make it clear that
the 50-byte input is interpreted as a 20x20 grid of bits. I call the two loop variables r and c (row and column).

The innermost loop is looping over the pairs in the array \code{\{\{1,0\}, \{0,1\}\}} which are clearly displacements (dr, dc)
in the row and column axes respectively. In the innermost loop, the function checks to see if \code{bit[r,c] ==
bit[r+dr, c+dc]}. If they are equal, a pair of long if chains follows; each chain ends in a block that calls \code{
sub401FD0}.

On first blush, this function looks complicated since it does a bunch of array lookups and recurses, but I
realized it was just recursively inlined. The original function definition resembles the following:

\begin{minted}{cpp}
int f(int **vec, int v) {
    int *x = &vec[0][v];
    if (v != *x) {
        *x = f(vec, *x);
    }
    return *x;
}
\end{minted}
  
The hint says that we are "checking black/white connected components in 2D grid of cells". Assuming the input represents a 20x20 grid of "black/white" cells, it is now clear that the triple loop is basically gathering the connected components of the input (connected horizontally or vertically), with the sub401FD0 function implementing the famous union-find algorithm used to group components together.

  \subsection{Second loop}

  In the second set of nested loops, the function checks each number in the map. If the number is nonzero, the
function checks if the connected component in the input is black (0) and if it has size equal to the map's
number. Thus, the map basically describes the size of the black regions of the input.

  \subsection{Third loop}

  For each nonzero element in the map, this loop inserts the ID of the corresponding black region into a \code{
std::set}. If the region is already inserted, this loop fails, so each black region must correspond to only one
nonzero map element.

  \subsection{Fourth loop}
  
  For each black region in the map, this loop checks to see if the ID is in the set, and fails if it is not. This
enforces that each nonzero map element corresponds to exactly one black region, and vice-versa.

  \subsection{Fifth loop}
  
  This loop iterates over all the white squares and checks to see whether they are all in the same connected
component. In other words, all white squares must be connected.

  \subsection{Sixth loop}

  This last loop rejects the input if there are four white squares in a 2x2 box anywhere in the input.

  It is this requirement that finally clues me into what's going on here: the map is a Nurikabe puzzle, a
Japanese puzzle that requires you to fill in a grid of black and white squares using exactly the same rules as
are being checked here.
  \subsection{Solving Nurikabe}
  Armed with this knowledge, I downloaded and tried a few different Nurikabe solvers. The "DotNet Nurikabe
Solver" solved the puzzle, producing the solution

\begin{minted}{text}
01000101000100100110
01011111111111111010
11101001010010001110
10100111010101110011
10011100111101011110
11110111010011100010
10001100110100111010
11100011011111000111
10111001100010100000
10000111011110111111
11111100110011000100
10010110001100111100
10101011101010100111
11101101011111100010
10101011100010111011
11011100011110001100
01001011110001110100
11111110100111001100
10101010111101110111
10101010010110010010
\end{minted}
  
  Encoding this solution yields the flag:\mintinline[breaklines]{text}|hitcon{7O^Im//SAofbOAmFFFS33AY.VF^S=d3YsIo*
  (AA//FIfDE"=ibiYAi/.ibo11V=-^+JO/Sb-im1si^-D}|
  
\end{document}
